\section{Введение}
  \subsection{Наименование программы}
    Наименование программы - “Система сбора и учета данных о состоянии дома”. Далее по тексту – Система.
  \subsection{Краткая характеристика области применения}
    Система должна применяться доме (или ином жилом помещении) заказчика.  
  \subsection{Условные обозначения и сокращения}
    БД – база данных;
    
    СУБД – Система управление базами данных;
    
    ТЗ – Техническое задание.
  \section{Основания для разработки}  
  \subsection{Основания для проведения разработки}
    Основанием для проведения разработки является выполнением курсовой работы по дисциплине “Организация Баз данных”.
  \subsection{Наименование и условные обозначения темы разработки}
    Наименование темы разработки – “Разработка базы данных для автоматической информационной системы сбора и учета данных о состоянии дома”.
\section{Назначение разработки}
  \subsection{Функциональное назначение}
    Функциональный назначением Системы является хранение и редактирование информации о комнатах, датчиках, расположенных в здании, значениях их показателей,  устройств для изменения значений датчиков,  жильцах дома и их предпочтениях.
  \subsection{Эксплуатационное назначение}
  
    Система должна эксплуатироваться жильцами дома.
\section{Требование к программе или программному изделию}
  \subsection{Требования к функциональным характеристикам}
    \subsubsection{Требования к составу выполняемых функций}
    
      Система должна обеспечить выполнение перечисленных ниже функций:
      \begin{enumerate}
        \item Должна поддерживать ввод, редактирование и удаление информации о комнатах в помещении, а именно:
          \begin{itemize}
            \item Название;
            \item Площадь пола в метрах квадратных;
            \item Высота в метрах;
          \end{itemize}
        \item Должна поддерживать ввод, редактирование и удаление информации о датчиках в помещении, а именно:
          \begin{itemize}
            \item Тип датчика,
            \item комната, к которой он относится
            \item минимально возможное значение
            \item максимально возможное значение
          \end{itemize}
          При этом параметр «тип датчика» может принимать одно из заранее оговоренных значений.
        \item Должна поддерживать ввод, редактирование и удаление информации о устройствах для изменения значений датчиков в помещении, а именно:
          \begin{itemize}
            \item Тип устройства
            \item Комната, в которой он находится
          \end{itemize}
          При этом тип устройства может принимать одно из заранее оговоренных значений.
        \item Должна поддерживать ввод, редактирование и удаление информации о жильцах помещения, а именно:
          \begin{itemize}
            \item Имя жильца 
            \item Фамилия жильца
            \item Возраст
            \item Пол
            \item Комната, в которой находится в данный момент
            \item Состояние
          \end{itemize}
          При этом состояние жильца приобретает  одно из заранее оговоренных значений
        \item Должна поддерживать возможность сбора статистики действий жильцов в доме. Элемент статистики содержит следующую информацию: 
            \begin{itemize}
              \item жилец;
	            \item тип действия;
	            \item комната: 
            \end{itemize}
        \item Должна поддерживаться возможность сбора, поиска и просмотра статистики о времени работы различных устройств. Каждый элемент статистики должен содержать следующую информацию:
          \begin{itemize}
            \item Устройство;
            \item Время начала работы (С точностью до секунды);
            \item Время окончания работы (С точностью до секунды);
          \end{itemize}          
          По окончании поиска должен быть предоставлен список элементов, удовлетворяющих параметру поиска.
        \item Должна поддерживать возможность сбора, поиска и просмотра статистики значений датчиков в разных комнатах. Каждый элемент статистики должен содержать следующую информацию:
          \begin{itemize}
            \item Датчик;
	          \item Комната;
	          \item Время
	          \item Значение;
          \end{itemize}
          Поиск может должен осуществляться по одному из указанных выше полей (кроме поля «время»), либо по типу датчика.
          
          По окончании поиска должен быть предоставлен список элементов, удовлетворяющих параметру поиска. 
      \end{enumerate}
    \subsubsection{Требования к организации входных данных}
      Типы значений входных данных программы должны соответствовать следующим требованиям:
          \begin{table}[h!]
            \centering
            \caption{Внешний вид “Комната”}
            \label{room:size}
            \begin{tabular}{|l|l|l|l|}
            \hline
            Наименование & Тип & Размер & Описание \\ \hline
            Номер комнаты & число & 8 & Не редактируемо (задаётся системой) \\ \hline
            Название  & строка & 30 & \\ \hline
            Площадь  & число & 32 &  \\ \hline
            Высота    & число & 32 & \\ \hline
            \end{tabular}
          \end{table}
          \begin{table}[h!]
            \centering
            \caption{Внешний вид “Датчик”}
            \label{sensor:size}
            \begin{tabular}{|l|l|l|l|}
            \hline
            Наименование & Тип & Размер & Описание \\ \hline
            Номер датчика & число & 8 & Не редактируемо (задаётся системой)\\ \hline
            Тип датчика  & значение из списка & 8 & \\ \hline
            Номер комнаты   & число & 8 & \\ \hline
            Минимальное значение & число & 32 & \\ \hline
            Максимальное значение & число & 32 & \\ \hline
            \end{tabular}
          \end{table}
          \begin{table}[h!]
            \centering
            \caption{Внешний вид “Корректирующее устройство”}
            \label{sensor:size}
            \begin{tabular}{|l|l|l|l|}
            \hline
            Наименование & Тип & Размер & Описание \\ \hline
            Номер устройства & число & 8 & Не редактируемо (задаётся системой) \\ \hline
            Тип устройства & значение из списка & 8 & \\ \hline
            Номер Комнаты & число & 8 & \\ \hline
            \end{tabular}
          \end{table}
          
          \begin{table}[h!]
            \centering
            \caption{Внешний вид “Жильцы”}
            \label{sensor:size}
            \begin{tabular}{|l|l|l|l|}
            \hline
            Наименование & Тип & Размер & Описание \\ \hline
            Номер жильца & число & 8 & \\ \hline
            Имя & строка & 30 & \\ \hline
            Фамилия & строка & 50 & \\ \hline
            Пол & символ & 1 & \\ \hline
            Возраст & число & 8 & \\ \hline
            Номер комнаты & число & 8 & \\ \hline
            \end{tabular}
          \end{table}
          \begin{table}[h!]
            \centering
            \caption{Внешний вид “Статистика Датчиков”}
            \label{room:size}
            \begin{tabular}{|l|l|l|l|}
            \hline
            Наименование & Тип & Размер & Описание \\ \hline
            Номер комнаты & число & 8 & \\ \hline
            Номер датчика & число & 8 & \\ \hline
            Время & время и дата & 32 & \\ \hline
            Значение & число & 32 & \\ \hline
            \end{tabular}
          \end{table}
          \begin{table}[h!]
            \centering
            \caption{Внешний вид “Статистика работы устройств”}
            \label{room:size}
            \begin{tabular}{|l|l|l|l|}
            \hline
            Наименование & Тип & Размер & Описание \\ \hline
            Номер устройства & число & 8 & \\ \hline
            Время начала & время и дата & 32 &          \\ \hline
            Время окончания & время и дата & 32 &          \\ \hline
            \end{tabular}
          \end{table}
          \begin{table}[h!]
            \centering
            \caption{Внешний вид “Статистика действий жильцов”}
            \label{room:size}
            \begin{tabular}{|l|l|l|l|}
            \hline
            Наименование & Тип & Размер & Описание \\ \hline
            Номер жильца & число & 8 & \\ \hline
            Тип действий & значение из списка & 8 & \\ \hline
            Номер комнаты & число & 8 & \\ \hline
            \end{tabular}
          \end{table} 
    \subsubsection{Требования к организации выходных данных}
      \begin{table}[h!]
      \centering
      \caption{Список значений датчиков}
      \label{sensor:statistic}
      \begin{tabular}{|l|l|l|}
      \hline
      Номер датчика & Номер комнаты & Значение \\ \hline
      \end{tabular}
      \end{table}
      
      \begin{table}[h!]
      \centering
      \caption{Список времени работы устройств}
      \label{device:statistic}
      \begin{tabular}{|l|l|l|l|}
      \hline
      Номер устройства & Тип устройства & Время начала & Время окончания \\ \hline
      \end{tabular}
      \end{table}
  \subsection{Требования к надежности}
    \subsubsection{Требования к обеспечению устойчивого (надёжного) функционирования программы}
      \begin{itemize}
        \item Организация бесперебойного питания серверного и коммуникационного оборудования;
        \item Использование лицензионного программного обеспечения (указанного в пункте 4.4);
        \item Регулярное выполнение рекомендаций Министерства труда и социального развития РФ, изложенных в Постановлении от 23.07.1998 «Об утверждении межотраслевых типовых норм времени на работы по сервисному обслуживанию ПЭВМ и оргтехники и сопровождению программных средств». А именно -  проведение ремонтно-профилактических работ устройств на которых установлена система;
        \item Регулярное выполнение требований ГОСТ 51188-98. «Защита информации. Испытания программных средств на наличие компьютерных вирусов». А именно – порядка и методов, проведения испытаний программных средства наличие компьютерных вирусов. 
    \end{itemize}
    \subsubsection{Время восстановления после отказа}
      Время восстановления состояния данных Системы после отказа, вызванного сбоем электропитания технических средств (иными внешними факторами), не фатальным сбоем операционной системы или файловой системы, не должно превышать 30 минут при соблюдении условий эксплуатации технических и программных средств и правильной настройке операционной системе.
    \subsubsection{Отказ из-за некорректных действий операторов}
      Возможными считаются отказы Системы (нарушение штатного режима функционирования) вследствие некоторых действий персонала, обслуживающего СУБД, операционную систему, под управлением которой работает Система. Защита от подобных действий настоящим Техническим заданием не предусматривается. Меры безопасности по недопущению некорректных действий персонала должны определяться соответствующими должностными инструкциями.
  \subsection{Условия эксплуатации}
    \subsubsection{Требования к климатическим условиям}
      Требования к климатическим условиям не предъявляются.
    \subsubsection{Требования к видам обслуживания}
      Обслуживание должно содержать информационное обслуживание – ввод и редактирование информации в БД.
    \subsubsection{Требования к численности и квалификации персонала}
      Для эксплуатации программы необходимо минимум 1 человека: Жилец дома .
Жилец дома должен иметь базовые навыки по использованию персонального компьютера, пройти курс обучения по использованию системы.
Возможно использование технического оператора, имеющего достаточную компетентность по управлению процессами, связанными с поддержанием комфортной среды в доме, для поддержания работоспособности Системы и перезапуска последней после возможных сбоев.
  \subsection{Требования к составу технических средств}
    \subsubsection{Сервер БД}
      Минимальные системные требования:
      \begin{itemize}
        \item Процессор x86 1.4 GHz или мощнее
        \item Оперативная память не менее 1 Gb
        \item Доступ к интернет сети со скоростью не менее 1 Мбит/сек. 
        \item Не менее 25 Gb свободного места жесткого диска.
      \end{itemize}
    \subsubsection{Клиент}
      Минимальные системные требования:
      \begin{itemize}
        \item Процессор x86 1.4 GHz или мощнее.
        \item Оперативная память не менее 2 Gb.
        \item Операционная система — любая, под которую доступен браузер Chrome.
        \item Жёсткий диск не менее 15 Gb свободного места, необходимого для установки операционной системы и браузера.
        \item Экран с разрешением не менее 640x480 пикселей.
        \item Манипуляторы ввода/вывода (клавиатура и мышь).
      \end{itemize}
  \subsection{Требования к программной и информационной совместимости}
    \subsubsection{Требования к информационным структурам и методам решения}
      Проектирование структуры БД должно быть выполнено в рамках разработки технического проекта.
    \subsubsection{Требования к исходным кодам и языкам программирования}
      Основным языком программирования разработки интерфейса БД должен быть язык программирования python3, в среде разработки PyCharm.
    \subsubsection{Требования к программным средствам. используемым программой}
      Наличие браузера Google Chrome с поддержкой HTML5, CSS3, ES6(JavaScript ).
    \subsubsection{Требования к защите информации и программ}
      Требования к БД на чтение и запись, осуществляется при наличие соответствующих прав доступа, для подтверждения прав доступа пользователю необходимо авторизоваться. Разграничение прав доступа должно быть реализовано средствами СУБД.
  \subsection{Требования к маркировке и упаковке}
    \subsubsection{Требования к маркировке}
      Требования к маркировке не предъявлются.
    \subsubsection{Требования к упаковке}
      Программа устанавливается на компьютерах заказчика, оригинальная копия программы предоставляется на внешнем носителе.
    \subsubsection{Условия упаковывания}
      Запись на внешний носитель должен проводится в закрытом помещении с относительной влажностью воздуха не более 80 процентов, при изоляции от агрессивных веществ.
  \subsection{Специальные требования}
    \subsubsection{Требования к пользовательскому интерфейсу}
      Особых требований к пользовательскому интерфейсу не предъявляются.
    \subsubsection{Требования к архивированию и резервному копированию}
      БД подлежит периодическому резервному копированию. Резервная копия Системы должна быть сделана единовременно после установки Системы и запуска её в эксплуатацию.
\section{Требования к программной документации}
  \subsection{Предварительные требования к программной документации}
    Состав программной документации должен включать в себя:
    \begin{enumerate}
    \item Техническое задание;
    \item Пояснительная записка;
    \item Руководство пользователя;
    \end{enumerate}
\section{Технико-экономические показатели}
  \subsection{Экономические преимущества разработки}
    Экономические преимущества разработки не рассчитываются 
\section{Стадии и этапы разработки}
  \subsection{Стадии разработки}
   \begin{enumerate}
    \item Разработка ТЗ
    \item Разработка БД
    \item Разработка Интерфейса БД
    \item Внедрение
  \end{enumerate}
  \subsection{Этапы разработки}
  \begin{itemize}
    \item На стадии разработки ТЗ должен быть выполнен этап разработки, согласования и утверждения настоящего ТЗ.
    \item На стадии разработки БД должен быть выполнен этап разработки БД.
    \item На стадии разработки интерфейса БД должен быть выполнен этап разработки интерфейса для доступа к информации в БД.
    \item На стадии внедрения должны быть выполнены подготовка и передача программы Заказчику.
  \end{itemize}
  
  \subsection{Содержание работ по этапам}
    На стадии разработки ТЗ должны быть выполнены перечисленные ниже работы:
      \begin{itemize}
        \item Постановка задачи;
        \item Определение и уточнение требований к техническим средствам;
        \item Определение требований к информационной системе;
        \item Определение стадий и этапов разработки информационной системы;
        \item Обоснование и выбор инструментария;
        \item Согласование и утверждение ТЗ.
      \end{itemize}
      
    На этапе разработки проектной документации должны быть выполнены перечисленные работы:
      \begin{itemize}
        \item Создание концептуальной модели БД;
        \item Создание физической модели БД;
      \end{itemize}
      
    На этапе разработки БД должна быть выполнена работа по разработке базы данных на основе технического задания и проектной документации.

    На этапе разработки интерфейса БД должна быть выполнена разработка интерфейсов базы данных.

    На этапе внедрения должна быть выполнена работа по подготовке и передаче программы и программной документации в эксплуатацию.

\section{Порядок контроля и приемки}
  \subsection{Виды испытаний}
    Приемосдаточные испытания должны проводиться на объекте Заказчика в сроки с _______ по _______.\\
    Приемосдаточные испытания программы должны проводиться согласно разработанной Исполнителем и согласованной с Заказчиком Программы методик испытаний.
  \subsection{Общие требования к приёмке работ}
    На основании Протокола проведения испытаний Исполнитель совместно с Заказчиком подписывает Акт приёма-сдачи программы в эксплуатацию.
\endinput
