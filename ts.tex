\section{Введение}
  \subsection{Наименование программы}
    Наименование программы - “Система сбора и учета данных о состоянии дома”. Далее по тексту – Система.
  \subsection{Краткая характеристика области применения}
    Система должна применяться доме (или ином жилом помещении) заказчика.  
  \subsection{Условные обозначения и сокращения}
    БД – база данных;
    
    СУБД – Система управление базами данных;
    
    ТЗ – Техническое задание.
  \section{Основания для разработки}  
  \subsection{Основания для проведения разработки}
    Основанием для проведения разработки является выполнением курсовой работы по дисциплине “Организация Баз данных”.
  \subsection{Наименование и условные обозначения темы разработки}
    Наименование темы разработки – “Разработка базы данных для информационной системы сбора и учета данных о состоянии дома”.
\section{Назначение разработки}
  \subsection{Функциональное назначение}
    Функциональным назначением Системы является хранение и редактирование информации о комнатах, датчиках, расположенных в здании, значениях их показателей,  устройств для изменения значений датчиков, жителях дома, их действиях и отзывах.
  \subsection{Эксплуатационное назначение}
  
    Система должна эксплуатироваться жителями дома.
\section{Требование к программе или программному изделию}
  \subsection{Требования к функциональным характеристикам}
    \subsubsection{Требования к составу выполняемых функций}
    
      Система должна обеспечить выполнение перечисленных ниже функций:
      \begin{enumerate}
        \item Должна поддерживать ввод, редактирование и удаление информации о комнатах в помещении, а именно:
          \begin{itemize}
            \item Название комнаты;
            \item Площадь пола в метрах квадратных;
            \item Высота в метрах;
          \end{itemize}
        \item Должна поддерживать ввод, редактирование и удаление информации о датчиках в помещении, а именно:
          \begin{itemize}
            \item Идентификатор типа датчика;
            \item Комната, к которой он относится;
            \item Минимально возможное значение;
            \item Максимально возможное значение;
          \end{itemize}
        \item Должна поддерживать ввод, редактирование и удаление информации о устройствах для изменения значений датчиков в помещении, а именно:
          \begin{itemize}
            \item Идентификатор типа устройства;
            \item Комната, в которой он находится;
          \end{itemize}
        \item Должна поддерживать ввод, редактирование и удаление информации о жителях здания, а именно:
          \begin{itemize}
            \item Имя жителя; 
            \item фамилия жителя;
            \item Возраст;
            \item Пол;
            \item Идентификатор комнаты, в которой находится в данный момент;
            \item Идентификатор состояния;
          \end{itemize}
        \item Должна поддерживать возможность сбора статистики действий жителей в доме. Элемент статистики должен содержать следующую информацию: 
            \begin{itemize}
              \item Идентификатор жителя;
	            \item Тип действия;
	            \item Идентификатор комнаты; 
            \end{itemize}
        \item Должна поддерживать возможность сбора отзывов жителей о значениях датчиков в комнатах. Отзыв должен содержать следующую  информацию: 
            \begin{itemize}
              \item Идентификатор жителя;
	            \item Тип датчика;
	            \item Значение датчика;
	            \item Идентификатор комнаты;
	            \item Доволен ли житель;
            \end{itemize}
        \item Должна поддерживаться возможность сбора, поиска и просмотра статистики о времени работы различных устройств. Каждый элемент статистики должен содержать следующую информацию:
          \begin{itemize}
            \item Идентификатор устройства;
            \item Время начала работы (С точностью до секунды);
            \item Время окончания работы (С точностью до секунды);
          \end{itemize}          
          По окончании поиска должен быть предоставлен список элементов, удовлетворяющих параметру поиска.
        \item Должна поддерживать возможность сбора, поиска и просмотра статистики значений датчиков в разных комнатах. Каждый элемент статистики должен содержать следующую информацию:
          \begin{itemize}
            \item Идентификатор датчика;
	          \item идентификатор комнаты;
	          \item Время получения значения;
	          \item Значение;
          \end{itemize}
          
          Поиск может должен осуществляться по одному из указанных выше полей (кроме поля «время»), либо по типу датчика.
          
          По окончании поиска должен быть предоставлен список элементов, удовлетворяющих параметру поиска. 
      \end{enumerate}
    \subsubsection{Требования к организации входных данных}
      Типы значений входных данных программы должны соответствовать следующим требованиям:
          \begin{table}[h!]
            \centering
            \caption{Внешний вид “Комната”}
            \label{room:size}
            \begin{tabular}{|l|l|l|l|}
            \hline
            Наименование & Тип & Размер & Описание \\ \hline
            Номер комнаты & число & 8 & Не редактируемо (задаётся системой) \\ \hline
            Название  & строка & 30 & \\ \hline
            Площадь  & число & 32 &  \\ \hline
            Высота    & число & 32 & \\ \hline
            \end{tabular}
          \end{table}
          \begin{table}[h!]
            \centering
            \caption{Внешний вид “Датчик”}
            \label{sensor:size}
            \begin{tabular}{|l|l|l|l|}
            \hline
            Наименование & Тип & Размер & Описание \\ \hline
            Номер датчика & число & 8 & Не редактируемо (задаётся системой)\\ \hline
            Тип & число & 8 & \\ \hline
            Номер комнаты   & число & 8 & \\ \hline
            Минимальное значение & число & 32 & \\ \hline
            Максимальное значение & число & 32 & \\ \hline
            \end{tabular}
          \end{table}
          \begin{table}[h!]
            \centering
            \caption{Внешний вид “Корректирующее устройство”}
            \label{sensor:size}
            \begin{tabular}{|l|l|l|l|}
            \hline
            Наименование & Тип & Размер & Описание \\ \hline
            Номер устройства & число & 8 & Не редактируемо (задаётся системой) \\ \hline
            Тип & число & 8 & \\ \hline
            Номер Комнаты & число & 8 & \\ \hline
            \end{tabular}
          \end{table}
          
          \begin{table}[h!]
            \centering
            \caption{Внешний вид “Жители здания”}
            \label{sensor:size}
            \begin{tabular}{|l|l|l|l|}
            \hline
            Наименование & Тип & Размер & Описание \\ \hline
            Номер жителя & число & 8 & \\ \hline
            Имя & строка & 30 & \\ \hline
            Фамилия & строка & 50 & \\ \hline
            Пол & символ & 1 & \\ \hline
            Возраст & число & 8 & \\ \hline
            Номер комнаты & число & 8 & \\ \hline
            \end{tabular}
          \end{table}
          \begin{table}[h!]
            \centering
            \caption{Внешний вид “Значения датчиков”}
            \label{room:size}
            \begin{tabular}{|l|l|l|l|}
            \hline
            Наименование & Тип & Размер & Описание \\ \hline
            Номер комнаты & число & 8 & \\ \hline
            Номер датчика & число & 8 & \\ \hline
            Время & время и дата & 32 & \\ \hline
            Значение & число & 32 & \\ \hline
            \end{tabular}
          \end{table}
          \begin{table}[h!]
            \centering
            \caption{Внешний вид “Время работы устройств”}
            \label{room:size}
            \begin{tabular}{|l|l|l|l|}
            \hline
            Наименование & Тип & Размер & Описание \\ \hline
            Номер устройства & число & 8 & \\ \hline
            Время начала & время и дата & 32 &          \\ \hline
            Время окончания & время и дата & 32 &          \\ \hline
            \end{tabular}
          \end{table}
          \begin{table}[h!]
            \centering
            \caption{Внешний вид “Действия жителей”}
            \label{room:size}
            \begin{tabular}{|l|l|l|l|}
            \hline
            Наименование & Тип & Размер & Описание \\ \hline
            Номер жителя & число & 8 & \\ \hline
            Идентификатор типа действий & значение из списка & 8 & \\ \hline
            Номер комнаты & число & 8 & \\ \hline
            \end{tabular}
          \end{table}
          \begin{table}[h!]
            \centering
            \caption{Внешний вид “Отзывы жителей”}
            \label{room:size}
            \begin{tabular}{|l|l|l|l|}
            \hline
            Наименование & Тип & Размер & Описание \\ \hline
            Номер жителя & число & 8 & \\ \hline
            Идентификатор типа датчика & число & 8 & \\ \hline
            Значение датчика & число & 32 & \\ \hline
            Удовлетворение жителя & ложь или истина & 1 & \\ \hline
            Номер комнаты & число & 8 & \\ \hline
            \end{tabular}
          \end{table} 
    \subsubsection{Требования к организации выходных данных}
      \begin{table}[H]
      \centering
      \caption{Список значений датчиков}
      \label{sensor:statistic}
      \begin{tabular}{|l|l|l|}
      \hline
      Номер датчика & Номер комнаты & Значение \\ \hline
      \end{tabular}
      \end{table}
      
      \begin{table}[H]
      \centering
      \caption{Список времени работы устройств}
      \label{device:statistic}
      \begin{tabular}{|l|l|l|l|}
      \hline
      Номер устройства & Тип устройства & Время начала & Время окончания \\ \hline
      \end{tabular}
      \end{table}
  \subsection{Требования к надежности}
    \subsubsection{Требования к обеспечению устойчивого (надёжного) функционирования программы}
      Надежное (устойчивое) функционирование программы должно быть обеспечено выполнением совокупности организационно-технических мероприятий, перечень которых приведен ниже:
      \begin{itemize}
        \item Организация бесперебойного питания серверного и коммуникационного оборудования;
        \item Использование лицензионного программного обеспечения (указанного в пункте 4.5.3);
      \end{itemize}
    \subsubsection{Время восстановления после мягкого сбоя}
      Время восстановления состояния данных Системы после отказа, вызванного не фатальным сбоем операционной системы или файловой системы, не должно превышать 30 минут при соблюдении условий эксплуатации технических и программных средств и правильной настройке операционной системы.
    \subsubsection{Время восстановления после жесткого сбоя}
      Время восстановления состояния данных Системы после отказа, вызванного сбоем электропитания технических средств (иными внешними факторами), фатальным сбоем операционной системы или файловой системы, не должно превышать времени, необходимого для восстановления нормальной работы операционной системы, файловой системы, а также запуска Системы при соблюдении условий эксплуатации технических и программных средств и правильной настройке операционной системы.
    \subsubsection{Отказ из-за некорректных действий операторов}
      Возможными считаются отказы Системы (нарушение штатного режима функционирования) вследствие некоторых действий персонала, обслуживающего СУБД, операционную систему, под управлением которой работает Система. Защита от подобных действий настоящим Техническим заданием не предусматривается. Меры безопасности по недопущению некорректных действий персонала должны определяться соответствующими должностными инструкциями.
  \subsection{Условия эксплуатации}
    \subsubsection{Требования к климатическим условиям}
      Требования к климатическим условиям не предъявляются.
    \subsubsection{Требования к видам обслуживания}
      Обслуживание должно содержать информационное обслуживание – ввод и редактирование информации в БД.
    \subsubsection{Требования к численности и квалификации персонала}
      Для эксплуатации программы необходим минимум 1 человек: житель дома .
Житель дома должен иметь базовые навыки по использованию персонального компьютера, пройти курс обучения по использованию системы.
Возможно использование технического оператора, имеющего достаточную компетентность по управлению процессами, связанными с поддержанием комфортной среды в доме, для поддержания работоспособности Системы и перезапуска последней после возможных сбоев.
  \subsection{Требования к составу технических средств}
    \subsubsection{Минимальные системные требования к серверу БД}
      \begin{itemize}
        \item Процессор x86 1.4 GHz или мощнее
        \item Оперативная память не менее 1 Gb
        \item Доступ к интернет сети со скоростью не менее 1 Мбит/сек. 
        \item Не менее 25 Gb свободного места жесткого диска.
      \end{itemize}
    \subsubsection{Минимальные системные требования к пользовательской ЭВМ}
      \begin{itemize}
        \item Процессор x86 1.4 GHz или мощнее.
        \item Оперативная память не менее 2 Gb.
        \item Операционная система — любая, под которую доступен браузер Chrome.
        %\item Наличие установленного браузера Chrome.
        \item Жёсткий диск не менее 15 Gb свободного места, необходимого для установки операционной системы и браузера.
        \item Экран с разрешением не менее 640x480 пикселей.
        \item Манипуляторы ввода/вывода (клавиатура и мышь).
      \end{itemize}
  \subsection{Требования к программной и информационной совместимости}
    \subsubsection{Требования к информационным структурам и методам решения}
      Проектирование структуры БД должно быть выполнено в рамках разработки технического проекта.
    \subsubsection{Требования к исходным кодам и языкам программирования}
      Основным языком программирования разработки интерфейса БД должен быть язык программирования python, в среде разработки PyCharm.
    \subsubsection{Требования к программным средствам. используемым программой}
    \begin{enumerate}
      \item Использование браузера Google Chrome с поддержкой HTML5, CSS3, ES6(JavaScript) на пользовательской ЭВМ ;
      \item Использование фреймворка Django и django-rest-framework на сервере; 
    \end{enumerate}
    \subsubsection{Требования к защите информации и программ}
      Требования к БД на чтение и запись реализуются при наличии соответствующих прав доступа, для подтверждения прав доступа пользователю необходимо авторизоваться.
  \subsection{Требования к маркировке и упаковке}
    \subsubsection{Требования к маркировке}
      Требования к маркировке не предъявляются.
    \subsubsection{Требования к упаковке}
      Программа устанавливается на компьютерах заказчика, оригинальная копия программы предоставляется на внешнем носителе.
    \subsubsection{Условия упаковывания}
      Запись на внешний носитель должен проводится в закрытом помещении с относительной влажностью воздуха не более 80 процентов, при изоляции от агрессивных веществ.
  \subsection{Специальные требования}
    \subsubsection{Требования к пользовательскому интерфейсу}
      Особых требований к пользовательскому интерфейсу не предъявляется.
    \subsubsection{Требования к архивированию и резервному копированию}
      БД подлежит периодическому резервному копированию. Резервная копия Системы должна быть сделана единовременно после установки Системы и запуска её в эксплуатацию.
\section{Требования к программной документации}
  \subsection{Предварительные требования к программной документации}
    Состав программной документации должен включать в себя:
    \begin{itemize}
    \item Техническое задание;
    \item Пояснительная записка;
    \item Руководство пользователя;
    \end{itemize}
\section{Технико-экономические показатели}
  \subsection{Экономические преимущества разработки}
    Экономические преимущества разработки не рассчитываются 
\section{Стадии и этапы разработки}
  \subsection{Стадии разработки}
   \begin{itemize}
    \item Разработка ТЗ
    \item Разработка БД
    \item Разработка пользовательского интерфейса
    \item Внедрение
  \end{itemize}
  \subsection{Этапы разработки}
  \begin{itemize}
    \item На стадии разработки ТЗ должен быть выполнен этап разработки, согласования и утверждения настоящего ТЗ.
    \item На стадии разработки БД должна быть выполнен этап разработки концептуальной и физической модель БД.
    \item На стадии разработки пользовательского интерфейса должен быть выполнен этап разработки интерфейса пользователя для доступа к информации в БД.
    \item На стадии внедрения должны быть выполнены подготовка и передача программы Заказчику.
  \end{itemize}
  
  \subsection{Содержание работ по этапам}
    На этапе разработки ТЗ должны быть выполнены перечисленные ниже работы:
      \begin{itemize}
        \item Постановка задачи;
        \item Определение и уточнение требований к техническим средствам;
        \item Определение требований к информационной системе;
        \item Определение стадий и этапов разработки информационной системы;
        \item Обоснование и выбор инструментария для разработки;
        \item Согласование и утверждение ТЗ.
      \end{itemize}
        
    На этапе разработки моделей БД должна быть выполнена работа по разработке концептуальной и физической модели базы данных на основе технического задания и проектной документации.

    На этапе разработки пользовательского интерфейса должна быть выполнена разработка графических интерфейсов пользователя для доступа к БД.

    На этапе внедрения должна быть выполнена работа по написанию пояснительной записки, руководства пользователя, подготовке и передаче программы и программной документации в эксплуатацию.
\section{Порядок контроля и приемки}
  \subsection{Виды испытаний}
    Приемосдаточные испытания должны проводиться на объекте Заказчика в сроки с \rule{2cm}{.1pt} по \rule{2cm}{.1pt}  .\\
    Приемосдаточные испытания программы должны проводиться согласно разработанной Исполнителем и согласованной с Заказчиком Программы методик испытаний.
  \subsection{Общие требования к приёмке работ}
    На основании Протокола проведения испытаний Исполнитель совместно с Заказчиком подписывает Акт приёма-сдачи программы в эксплуатацию.
\endinput
